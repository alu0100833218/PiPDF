\documentclass[spanish,a4paper,10pt]{article}

\usepackage{latexsym,amsfonts,amssymb,amstext,amsthm,float,amsmath}
\usepackage[spanish]{babel}
\usepackage[latin1]{inputenc}
\usepackage[dvips]{epsfig}
\usepackage{doc}

%%%%%%%%%%%%%%%%%%%%%%%%%%%%%%%%%%%%%%%%%%%%%%%%%%%%%%%%%%%%%%%%%%%%%%
%123456789012345678901234567890123456789012345678901234567890123456789
%%%%%%%%%%%%%%%%%%%%%%%%%%%%%%%%%%%%%%%%%%%%%%%%%%%%%%%%%%%%%%%%%%%%%%
%\textheight    29cm
%\textwidth     15cm
%\topmargin     -4cm
%\oddsidemargin  5mm

%%%%%%%%%%%%%%%%%%%%%%%%%%%%%%%%%%%%%%%%%%%%%%%%%%%%%%%%%%%%%%%%%%%%%%
\begin{figure}[t]
\begin{center}
\includegraphics[scale=0.1]{src/imagen1.eps}
\end{center}
\end{figure}

\begin{document}


\title{El n�mero $\pi$}
\author{ David Tom�s Montesdeoca Flores \\ Pr�ctica de Laboratorio \#10}
\date{9 de abril de 2014}

\maketitle

\begin{abstract}
El objetivo de este informe es explicar el n�mero $\pi$. 
\end{abstract}

%\thispagestyle{empty}
%++++++++++++++++++++++++++++++++++++++++++++++++++++++++++++++++++++++
\section{Definici�n}

El n�mero $\pi$ es la relaci�n existente entre el di�metro de la circunferencia con su longitud.
Es un n�mero irracional de los m�s importantes usados en las ciencias matem�ticas, como la f�sica, las ingenier�as y las propias matem�ticas.
%
El valor que toma esta constante es aproximadamente:
 $$\pi = 3.14159265358979323846...$$ 
%
Como hemos visto en pr�cticas anteriores, este se puede calcular mediante integraci�n:

$$\int_{0}^{1} \! \frac{4}{1+x^2}\, dx = 4(atan(1) -atan(0)) = \pi $$



%++++++++++++++++++++++++++++++++++++++++++++++++++++++++++++++++++++++
\section{Historia del C�lculo del n�mero \pi}

El c�lculo del n�mero $\pi$ a lo largo de la historia ha sido una ardua tarea para los cientificos que han llevado a cabo sus aproximaciones.
%
Algunas de sus aproximaciones a lo largo de la historia m�s importantes han tenido lugar en:

\begin{enumerate}
  \item
    El Antiguo Egipto.
  \item
    La Antig�edad Cl�sica (Grecia y Roma).
  \item
    Mesopotamia.
  \item
    La India.
  \item
    China
  \item
    Europa (Durante el Renacimiento)
  \item
    Persia
\end{enumerate}

En la �poca actual el mayor numero de decimales obtenido se llev�  a cabo por Shigeru Kondo, obteniendo 10.000.000.000.000 cifras.

\begin{table}{}
\begin{tabular}{lrcl}
A�o   &  Nombre       &  Ordenador &  N�mero de decimales \\ \hline
1949  &  Reitwiesner  &  ENIAC     &  2.037  \\ \hline
1959  &  Guilloud     &  IBM 704   &  16.167  \\ \hline
1986  &  Bailey       &  CRAY-2    &  29.360.111 \\ \hline
2011  &  Kondo        &            &  10.000.000.000.000 \\ \hline
\end{tabular}
\end {table}





%++++++++++++++++++++++++++++++++++++++++++++++++++++++++++++++++++++++
\section{Algunas F�rmulas que contienen el numero \pi}
\subsection{Geometr�a} 
\begin{enumerate}
  \item
    Longitud de la circunferencia.
  \item
    �rea del c�rculo.
  \item
    �rea interior de la elipse.
  \item
    �rea del cono.
  \item
    �rea de la esfera.
\end {enumerate}

\subsection{An�lisis} 
\begin{enumerate}
  \item
    F�rmula de Leibniz.
  \item
    Producto de Wallis.
  \item
    F�rmula de Euler.
  \item
    F�rmula de Stirling.
  \item
    M�todo de Montecarlo
\end{enumerate}
\subsection{C�lculo} 
\begin{enumerate}
  \item
   �rea limitada por la astroide: $\frac{3}{8}\pi a^2 $.\footnote{Tipo de curva con cuatro v�rtices.}

  \item
    �rea de la regi�n comprendida por el eje X y un arco de la cicloide: $3 \pi a^2.$\footnote {curva generada por un punto perteneciente a una circunferencia generatriz al rodar sobre una l�nea recta directriz, sin deslizarse.}
\end{enumerate}
%++++++++++++++++++++++++++++++++++++++++++++++++++++++++++++++++++++++
\section{Bibliograf�a}
\begin{itemize}
  \bibitem
  es.wikipedia.org/wiki/N�mero\_\pi \hline 
  \bibitem
     www.juegosdelogica.com/numero\_\pi.htm
\end{itemize}
\end{document}